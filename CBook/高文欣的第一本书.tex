% Options for packages loaded elsewhere
\PassOptionsToPackage{unicode}{hyperref}
\PassOptionsToPackage{hyphens}{url}
%
\documentclass[
]{book}
\usepackage{lmodern}
\usepackage{amssymb,amsmath}
\usepackage{ifxetex,ifluatex}
\ifnum 0\ifxetex 1\fi\ifluatex 1\fi=0 % if pdftex
  \usepackage[T1]{fontenc}
  \usepackage[utf8]{inputenc}
  \usepackage{textcomp} % provide euro and other symbols
\else % if luatex or xetex
  \usepackage{unicode-math}
  \defaultfontfeatures{Scale=MatchLowercase}
  \defaultfontfeatures[\rmfamily]{Ligatures=TeX,Scale=1}
\fi
% Use upquote if available, for straight quotes in verbatim environments
\IfFileExists{upquote.sty}{\usepackage{upquote}}{}
\IfFileExists{microtype.sty}{% use microtype if available
  \usepackage[]{microtype}
  \UseMicrotypeSet[protrusion]{basicmath} % disable protrusion for tt fonts
}{}
\makeatletter
\@ifundefined{KOMAClassName}{% if non-KOMA class
  \IfFileExists{parskip.sty}{%
    \usepackage{parskip}
  }{% else
    \setlength{\parindent}{0pt}
    \setlength{\parskip}{6pt plus 2pt minus 1pt}}
}{% if KOMA class
  \KOMAoptions{parskip=half}}
\makeatother
\usepackage{xcolor}
\IfFileExists{xurl.sty}{\usepackage{xurl}}{} % add URL line breaks if available
\IfFileExists{bookmark.sty}{\usepackage{bookmark}}{\usepackage{hyperref}}
\hypersetup{
  pdftitle={gaowenxin的第一本书},
  pdfauthor={高文欣},
  hidelinks,
  pdfcreator={LaTeX via pandoc}}
\urlstyle{same} % disable monospaced font for URLs
\usepackage{color}
\usepackage{fancyvrb}
\newcommand{\VerbBar}{|}
\newcommand{\VERB}{\Verb[commandchars=\\\{\}]}
\DefineVerbatimEnvironment{Highlighting}{Verbatim}{commandchars=\\\{\}}
% Add ',fontsize=\small' for more characters per line
\usepackage{framed}
\definecolor{shadecolor}{RGB}{248,248,248}
\newenvironment{Shaded}{\begin{snugshade}}{\end{snugshade}}
\newcommand{\AlertTok}[1]{\textcolor[rgb]{0.94,0.16,0.16}{#1}}
\newcommand{\AnnotationTok}[1]{\textcolor[rgb]{0.56,0.35,0.01}{\textbf{\textit{#1}}}}
\newcommand{\AttributeTok}[1]{\textcolor[rgb]{0.77,0.63,0.00}{#1}}
\newcommand{\BaseNTok}[1]{\textcolor[rgb]{0.00,0.00,0.81}{#1}}
\newcommand{\BuiltInTok}[1]{#1}
\newcommand{\CharTok}[1]{\textcolor[rgb]{0.31,0.60,0.02}{#1}}
\newcommand{\CommentTok}[1]{\textcolor[rgb]{0.56,0.35,0.01}{\textit{#1}}}
\newcommand{\CommentVarTok}[1]{\textcolor[rgb]{0.56,0.35,0.01}{\textbf{\textit{#1}}}}
\newcommand{\ConstantTok}[1]{\textcolor[rgb]{0.00,0.00,0.00}{#1}}
\newcommand{\ControlFlowTok}[1]{\textcolor[rgb]{0.13,0.29,0.53}{\textbf{#1}}}
\newcommand{\DataTypeTok}[1]{\textcolor[rgb]{0.13,0.29,0.53}{#1}}
\newcommand{\DecValTok}[1]{\textcolor[rgb]{0.00,0.00,0.81}{#1}}
\newcommand{\DocumentationTok}[1]{\textcolor[rgb]{0.56,0.35,0.01}{\textbf{\textit{#1}}}}
\newcommand{\ErrorTok}[1]{\textcolor[rgb]{0.64,0.00,0.00}{\textbf{#1}}}
\newcommand{\ExtensionTok}[1]{#1}
\newcommand{\FloatTok}[1]{\textcolor[rgb]{0.00,0.00,0.81}{#1}}
\newcommand{\FunctionTok}[1]{\textcolor[rgb]{0.00,0.00,0.00}{#1}}
\newcommand{\ImportTok}[1]{#1}
\newcommand{\InformationTok}[1]{\textcolor[rgb]{0.56,0.35,0.01}{\textbf{\textit{#1}}}}
\newcommand{\KeywordTok}[1]{\textcolor[rgb]{0.13,0.29,0.53}{\textbf{#1}}}
\newcommand{\NormalTok}[1]{#1}
\newcommand{\OperatorTok}[1]{\textcolor[rgb]{0.81,0.36,0.00}{\textbf{#1}}}
\newcommand{\OtherTok}[1]{\textcolor[rgb]{0.56,0.35,0.01}{#1}}
\newcommand{\PreprocessorTok}[1]{\textcolor[rgb]{0.56,0.35,0.01}{\textit{#1}}}
\newcommand{\RegionMarkerTok}[1]{#1}
\newcommand{\SpecialCharTok}[1]{\textcolor[rgb]{0.00,0.00,0.00}{#1}}
\newcommand{\SpecialStringTok}[1]{\textcolor[rgb]{0.31,0.60,0.02}{#1}}
\newcommand{\StringTok}[1]{\textcolor[rgb]{0.31,0.60,0.02}{#1}}
\newcommand{\VariableTok}[1]{\textcolor[rgb]{0.00,0.00,0.00}{#1}}
\newcommand{\VerbatimStringTok}[1]{\textcolor[rgb]{0.31,0.60,0.02}{#1}}
\newcommand{\WarningTok}[1]{\textcolor[rgb]{0.56,0.35,0.01}{\textbf{\textit{#1}}}}
\usepackage{longtable,booktabs}
% Correct order of tables after \paragraph or \subparagraph
\usepackage{etoolbox}
\makeatletter
\patchcmd\longtable{\par}{\if@noskipsec\mbox{}\fi\par}{}{}
\makeatother
% Allow footnotes in longtable head/foot
\IfFileExists{footnotehyper.sty}{\usepackage{footnotehyper}}{\usepackage{footnote}}
\makesavenoteenv{longtable}
\usepackage{graphicx,grffile}
\makeatletter
\def\maxwidth{\ifdim\Gin@nat@width>\linewidth\linewidth\else\Gin@nat@width\fi}
\def\maxheight{\ifdim\Gin@nat@height>\textheight\textheight\else\Gin@nat@height\fi}
\makeatother
% Scale images if necessary, so that they will not overflow the page
% margins by default, and it is still possible to overwrite the defaults
% using explicit options in \includegraphics[width, height, ...]{}
\setkeys{Gin}{width=\maxwidth,height=\maxheight,keepaspectratio}
% Set default figure placement to htbp
\makeatletter
\def\fps@figure{htbp}
\makeatother
\setlength{\emergencystretch}{3em} % prevent overfull lines
\providecommand{\tightlist}{%
  \setlength{\itemsep}{0pt}\setlength{\parskip}{0pt}}
\setcounter{secnumdepth}{5}
\usepackage{ctex}

%\usepackage{xltxtra} % XeLaTeX的一些额外符号
% 设置中文字体
\setCJKmainfont[BoldFont={黑体},ItalicFont={楷体}]{新宋体}

\usepackage{amsthm,mathrsfs}
\usepackage{booktabs}
\usepackage{longtable}
\makeatletter
\def\thm@space@setup{%
  \thm@preskip=8pt plus 2pt minus 4pt
  \thm@postskip=\thm@preskip
}
\makeatother
\usepackage[]{natbib}
\bibliographystyle{apalike}

\title{gaowenxin的第一本书}
\author{高文欣}
\date{2020-02-01}

\begin{document}
\maketitle

{
\setcounter{tocdepth}{1}
\tableofcontents
}
\hypertarget{ux7b80ux4ecb}{%
\chapter*{简介}\label{ux7b80ux4ecb}}
\addcontentsline{toc}{chapter}{简介}

R软件的bookdown扩展包是R Markdown的增强版,
支持自动目录、文献索引、公式编号与引用、定理编号与引用、图表自动编号与引用等功能,
可以作为LaTeX的一种替代解决方案,
在制作用R进行数据分析建模的技术报告时,
可以将报告文字、R程序、文字性结果、表格、图形都自动地融合在最后形成的网页或者PDF文件中。

Bookdown使用的设置比较复杂,
对初学者不够友好。
这里制作了一些模板,
用户只要解压缩打包的文件,
对某个模板进行修改填充就可以变成自己的中文图书或者论文。
Bookdown的详细用法参见\url{https://bookdown.org/yihui/bookdown/},
在李东风的\href{http://www.math.pku.edu.cn/teachers/lidf/docs/Rbook/html/_Rbook/index.html}{《统计软件教程》}也有部分介绍。

一些常用功能的示例在\texttt{0101-usage.Rmd}文件中,
用户可以在编辑器中打开此文件参考其中的做法。

Bookdown如果输出为网页,
其中的数学公式需要MathJax程序库的支持,
用如下数学公式测试浏览器中数学公式显示是否正常:

\[
\text{定积分} = \int_a^b f(x) \,dx
\]

如果显示不正常,
可以在公式上右键单击,
选择``Math Settings--Math Renderer'',
依次使用改成``Common HTML'',``SVG''等是否可以变成正常显示。
PDF版本不存在这样的问题。

\hypertarget{usage}{%
\chapter{中文图书Bookdown模板的基本用法}\label{usage}}

\hypertarget{usage-ins}{%
\section{安装设置}\label{usage-ins}}

使用RStudio软件完成编辑和转换功能。
在RStudio中,安装bookdown等必要的扩展包。

本模板在安装之前是一个打包的zip文件,
在适当位置解压(例如,在\texttt{C:/myproj}下),
得到\texttt{MathJax}, \texttt{Books/Cbook}, \texttt{Books/Carticle}等子目录。
本模板在\texttt{Books/Cbook}中。

为了利用模板制作自己的中文书,
将\texttt{Books/Cbook}制作一个副本,
改成适当的子目录名,如\texttt{Books/Mybook}。

打开RStudio软件,
选选单``File - New Project - Existing Directory'',
选中\texttt{Books/Mybook}子目录,确定。
这样生成一本书对应的R project(项目)。

为了将模板内容替换成自己的内容,
可以删除文件\texttt{0101-usage.Rmd},
然后将\texttt{1001-chapter01.Rmd}制作几份副本,
如\texttt{1001-chapter01.Rmd}, \texttt{2012-chapter02.Rmd},
\texttt{2012-chapter03.Rmd}。
各章的次序将按照前面的数值的次序排列。
将每个\texttt{.Rmd}文件内的\texttt{\{\#chapter01\}}, \texttt{\{\#chapter02-sec01\}}修改能够反映章节内容的标签文本。
所有的标签都不允许重复。
参见本模板中的\texttt{0101-usage.Rmd}文件。

后面的§\ref{usage-gitbook} 和§\ref{usage-pdfbook} 给出了将当前的书转换为网页和PDF的命令,
复制粘贴这些命令到RStudio命令行可以进行转换。

\hypertarget{usage-writing}{%
\section{编写自己的内容}\label{usage-writing}}

\hypertarget{usage-writing-struct}{%
\subsection{文档结构}\label{usage-writing-struct}}

除了\texttt{index.Rmd}以外,
每个\texttt{.Rmd}文件是书的一章。
每章的第一行是用一个井号(\texttt{\#})引入的章标题。
节标题用两个井号开始,
小节标题用三个井号开始。
标题后面都有大括号内以井号开头的标签,
标签仅用英文大小写字母和减号。

\hypertarget{usage-writing-fig}{%
\subsection{图形自动编号}\label{usage-writing-fig}}

用R代码段生成的图形,
只要具有代码段标签,
且提供代码段选项\texttt{fig.cap="图形的说明文字"},
就可以对图形自动编号,
并且可以用如\texttt{\textbackslash{}@ref(fig:label)}的格式引用图形。
如:

\begin{Shaded}
\begin{Highlighting}[]
\KeywordTok{plot}\NormalTok{(}\DecValTok{1}\OperatorTok{:}\DecValTok{10}\NormalTok{, }\DataTypeTok{main=}\StringTok{"程序生成的测试图形"}\NormalTok{)}
\end{Highlighting}
\end{Shaded}

\begin{figure}
\centering
\includegraphics{0101-usage_files/figure-latex/u-w-f-ex01-1.pdf}
\caption{\label{fig:u-w-f-ex01}图形说明文字}
\end{figure}

引用如:参见图\ref{fig:u-w-f-ex01}。
引用中的\texttt{fig:}是必须的。

在通过LaTeX转换的PDF结果中,
这样图形是浮动的。

\hypertarget{usage-writing-tab}{%
\subsection{表格自动编号}\label{usage-writing-tab}}

用R代码\texttt{knitr::kable()}生成的表格,
只要具有代码段标签,
并且在\texttt{knitr::kable()}调用时加选项\texttt{caption="表格的说明文字"},
就可以对表格自动编号,
并且可以用如\texttt{\textbackslash{}@ref(tab:label)}的格式引用表格。
如:

\begin{Shaded}
\begin{Highlighting}[]
\NormalTok{d <-}\StringTok{ }\KeywordTok{data.frame}\NormalTok{(}\StringTok{"自变量"}\NormalTok{=}\DecValTok{1}\OperatorTok{:}\DecValTok{10}\NormalTok{, }\StringTok{"因变量"}\NormalTok{=(}\DecValTok{1}\OperatorTok{:}\DecValTok{10}\NormalTok{)}\OperatorTok{^}\DecValTok{2}\NormalTok{)}
\NormalTok{knitr}\OperatorTok{::}\KeywordTok{kable}\NormalTok{(d, }\DataTypeTok{caption=}\StringTok{"表格说明文字"}\NormalTok{)}
\end{Highlighting}
\end{Shaded}

\begin{table}

\caption{\label{tab:u-w-tab-ex01}表格说明文字}
\centering
\begin{tabular}[t]{r|r}
\hline
自变量 & 因变量\\
\hline
1 & 1\\
\hline
2 & 4\\
\hline
3 & 9\\
\hline
4 & 16\\
\hline
5 & 25\\
\hline
6 & 36\\
\hline
7 & 49\\
\hline
8 & 64\\
\hline
9 & 81\\
\hline
10 & 100\\
\hline
\end{tabular}
\end{table}

引用如:参见表\ref{tab:u-w-tab-ex01}。
引用中的\texttt{tab:}是必须的。

在通过LaTeX转换的PDF结果中,
这样的表格是浮动的。

\hypertarget{usage-writing-math}{%
\subsection{数学公式编号}\label{usage-writing-math}}

不需要编号的公式,
仍可以按照一般的Rmd文件中公式的做法。
需要编号的公式,
直接写在\texttt{\textbackslash{}begin\{align\}}和\texttt{\textbackslash{}end\{align\}}之间,
不需要编号的行在末尾用\texttt{\textbackslash{}nonumber}标注。
需要编号的行用\texttt{(\textbackslash{}\#eq:mylabel)}添加自定义标签,
如

\begin{align}
\Sigma =&  (\sigma_{ij})_{n\times n} \nonumber \\
=& E[(\boldsymbol{X} - \boldsymbol{\mu}) (\boldsymbol{X} - \boldsymbol{\mu})^T ] 
\label{eq:var-mat-def}
\end{align}

引用如:协方差定义见\eqref{eq:var-mat-def}。

\hypertarget{ux6587ux732eux5f15ux7528ux4e0eux6587ux732eux5217ux8868}{%
\subsection{文献引用与文献列表}\label{ux6587ux732eux5f15ux7528ux4e0eux6587ux732eux5217ux8868}}

将所有文献用bib格式保存为一个\texttt{.bib}文献库,
如模板中的样例文件\texttt{mybib.bib}。
可以用JabRef软件来管理这样的文献库,
许多其它软件都可以输出这样格式的文件库。

为了引用某一本书,
用如:参见\citep{Wichmann1982:RNG}。

被引用的文献将出现在一章末尾以及全书的末尾,
对PDF输出则仅出现在全书末尾。

\hypertarget{usage-output}{%
\section{转换}\label{usage-output}}

\hypertarget{usage-gitbook}{%
\subsection{转换为网页}\label{usage-gitbook}}

用如下命令将整本书转换成一个每章为一个页面的网站,
称为gitbook格式:

\begin{Shaded}
\begin{Highlighting}[]
\NormalTok{bookdown}\OperatorTok{::}\KeywordTok{render_book}\NormalTok{(}\StringTok{"index.Rmd"}\NormalTok{, }
  \DataTypeTok{output_format=}\StringTok{"bookdown::gitbook"}\NormalTok{, }\DataTypeTok{encoding=}\StringTok{"UTF-8"}\NormalTok{)}
\end{Highlighting}
\end{Shaded}

为查看结果,
在\texttt{\_book}子目录中双击其中的\texttt{index.html}文件,
就可以在网络浏览器中查看转换的结果。
重新编译后应该点击``刷新''图标。

在章节和内容较多时,
通常不希望每次小修改之后重新编译整本书,
这时类似如下的命令可以仅编译一章,
可以节省时间,
缺点是导航目录会变得不准确。
命令如:

\begin{Shaded}
\begin{Highlighting}[]
\NormalTok{bookdown}\OperatorTok{::}\KeywordTok{preview_chapter}\NormalTok{(}\StringTok{"1001-chapter01.Rmd"}\NormalTok{,}
  \DataTypeTok{output_format=}\StringTok{"bookdown::gitbook"}\NormalTok{, }\DataTypeTok{encoding=}\StringTok{"UTF-8"}\NormalTok{)}
\end{Highlighting}
\end{Shaded}

单章的网页可以通过网络浏览器中的``打印''功能,
选择一个打印到PDF的打印机,
可以将单章转换成PDF格式。

\hypertarget{usage-pdfbook}{%
\subsection{生成PDF}\label{usage-pdfbook}}

Bookdown借助操作系统中另行安装的LaTeX编译软件将整本书转换成一个PDF文件。
这需要用户对LaTeX有一定的了解,
否则一旦出错,
就完全不知道如何解决。
用户如果需要进行LaTeX定制,
可修改模板中的\texttt{preamble.tex}文件。

转换为PDF的命令如下:

\begin{Shaded}
\begin{Highlighting}[]
\NormalTok{bookdown}\OperatorTok{::}\KeywordTok{render_book}\NormalTok{(}\StringTok{"index.Rmd"}\NormalTok{, }\DataTypeTok{output_format=}\StringTok{"bookdown::pdf_book"}\NormalTok{, }\DataTypeTok{encoding=}\StringTok{"UTF-8"}\NormalTok{)}
\end{Highlighting}
\end{Shaded}

在\texttt{\_book}子目录中找到\texttt{CBook.pdf}文件,
这是转换的结果。

转换PDF对于内容多的书比较耗时,
不要过于频繁地转换PDF,
在修改书的内容时,
多用\texttt{bookdown::preview\_chapter}和转换为gitbook的办法检验结果。
定期地进行转换PDF的测试。
每增加一章后都应该试着转换成PDF看有没有错误。

在MS Windows操作系统中,
可以安装MikTeX软件将LaTeX文件编译为PDF。
Bookdown需要一个这样的软件转换整本书为一个PDF文件。

\hypertarget{usage-website}{%
\subsection{上传到网站}\label{usage-website}}

如果书里面没有数学公式,
则上传到网站就只要将\texttt{\_book}子目录整个地用ftp软件传送到自己的网站主目录下的某个子目录即可。
但是,为了支持数学公式,就需要进行如下的目录结构设置:

\begin{enumerate}
\def\labelenumi{\arabic{enumi}.}
\tightlist
\item
  设自己的网站服务器目录为\texttt{/home/abc},
  将MathJax目录上传到这个目录中。
\item
  在\texttt{/home/abc}中建立新目录\texttt{Books/Mybook}。
\item
  将\texttt{\_book}子目录上传到\texttt{/home/abc/Books/Mybook}中。
\item
  这时网站链接可能类似于\texttt{http://dept.univ.edu.cn/\textasciitilde{}abc/Books/Mybooks/\_book/index.html},
  具体链接地址依赖于服务器名称与主页所在的主目录名称。
\end{enumerate}

如果有多本书,
\texttt{MathJax}仅需要上传一次。
因为\texttt{MathJax}有三万多个文件,
所以上传\texttt{MathJax}会花费很长时间。

\hypertarget{chapter-example}{%
\chapter{样例章}\label{chapter-example}}

\hypertarget{example-sectionex}{%
\section{第一章第一节}\label{example-sectionex}}

  \bibliography{mybib.bib}

\end{document}
